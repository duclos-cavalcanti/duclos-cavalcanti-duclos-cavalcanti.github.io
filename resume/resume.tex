\documentclass[10pt,a4paper,ragged2e,withhyper]{altacv}
%% AltaCV uses the fontawesome5 and packages.
%% See http://texdoc.net/pkg/fontawesome5 for full list of symbols.


% Change the page layout if you need to
\geometry{left=1.25cm,right=1.25cm,top=1.5cm,bottom=1.5cm,columnsep=1.2cm}

% The paracol package lets you typeset columns of text in parallel
\usepackage{paracol}

% Change the font if you want to, depending on whether
% you're using pdflatex or xelatex/lualatex
\ifxetexorluatex
  % If using xelatex or lualatex:
  \setmainfont{Roboto Slab}
  \setsansfont{Lato}
  \renewcommand{\familydefault}{\sfdefault}
\else
  % If using pdflatex:
  \usepackage[rm]{roboto}
  \usepackage[defaultsans]{lato}
  % \usepackage{sourcesanspro}
  \renewcommand{\familydefault}{\sfdefault}
\fi

% Change the colours if you want to
\definecolor{SlateGrey}{HTML}{2E2E2E}
\definecolor{LightGrey}{HTML}{666666}
\definecolor{TealGreen}{HTML}{5BC8AF}
\colorlet{name}{black}
\colorlet{tagline}{black}
\colorlet{heading}{black}
\colorlet{headingrule}{gray}
\colorlet{subheading}{blue}
\colorlet{accent}{black} % TealGreen
\colorlet{emphasis}{SlateGrey}
\colorlet{body}{LightGrey}

% Change some fonts, if necessary
\renewcommand{\namefont}{\Huge\rmfamily\bfseries}
\renewcommand{\personalinfofont}{\footnotesize}
\renewcommand{\cvsectionfont}{\LARGE\rmfamily\bfseries}
\renewcommand{\cvsubsectionfont}{\large\bfseries}


% Change the bullets for itemize and rating marker
% for \cvskill if you want to
\renewcommand{\cvItemMarker}{{\small\textbullet}}
\renewcommand{\cvRatingMarker}{\faCircle}
% ...and the markers for the date/location for \cvevent
% \renewcommand{\cvDateMarker}{\faCalendar*[regular]}
% \renewcommand{\cvLocationMarker}{\faMapMarker*}

% If your CV/résumé is in a language other than English,
% then you probably want to change these so that when you
% copy-paste from the PDF or run pdftotext, the location
% and date marker icons for \cvevent will paste as correct
% translations. For example Spanish:
% \renewcommand{\locationname}{Ubicación}
% \renewcommand{\datename}{Fecha}

\begin{document}
\name{Daniel Duclos-Cavalcanti}
\tagline{Dual Citizenship: U.S. \& Brazil -- Computer Engineer \hfill Languages: \textcolor{gray}{\textit{English, German, Portuguese}} }

\personalinfo{%
  % Not all of these are required!
  \homepage{www.duclos.dev}
  \email{me@duclos.dev}
  \phone{+1-516-912-7975}
  \location{New York, NY, USA}
  \github{duclos-cavalcanti}
  %\linkedin{daniel-duclos-cavalcanti}
  %% You can add your own arbitrary detail with
  %% \printinfo{symbol}{detail}[optional hyperlink prefix]
  % \printinfo{\faPaw}{Hey ho!}[https://example.com/]
  %% Or you can declare your own field with
  %% \NewInfoFiled{fieldname}{symbol}[optional hyperlink prefix] and use it:
  % \NewInfoField{gitlab}{\faGitlab}[https://gitlab.com/]
  % \gitlab{your_id}
  %%
  %% For services and platforms like Mastodon where there isn't a
  %% straightforward relation between the user ID/nickname and the hyperlink,
  %% you can use \printinfo directly e.g.
  % \printinfo{\faMastodon}{@username@instace}[https://instance.url/@username]
  %% But if you absolutely want to create new dedicated info fields for
  %% such platforms, then use \NewInfoField* with a star:
  % \NewInfoField*{mastodon}{\faMastodon}
  %% then you can use \mastodon, with TWO arguments where the 2nd argument is
  %% the full hyperlink.
  % \mastodon{@username@instance}{https://instance.url/@username}
}

\makecvheader
%% Depending on your tastes, you may want to make fonts of itemize environments slightly smaller
% \AtBeginEnvironment{itemize}{\small}

%% Set the left/right column width ratio to 6:4.
\columnratio{0.6}

% Start a 2-column paracol. Both the left and right columns will automatically
% break across pages if things get too long.
\begin{paracol}{2}
\cvsection{Education}

    \cvevent{Non-Degree Graduate Student}
        {New York University, Courant Institute of Mathematical Sciences}
        {Sept 2023 -- May 2024}{New York, USA}
\begin{itemize}
\item CSCI-GA 2250 - Operating Systems - \textbf{A}
\item CSCI-GA 3033 - Technologies in Finance - \textbf{Pass}
\item Collaboration: \href{https://arxiv.org/abs/2402.09527}{{\textcolor{black}{\underline{Jasper: Fair Multicast for Financial Exch. in the Cloud}}}}
\item Research Work Co-advised by:
    \begin{itemize}
        \item \href{https://anirudhsk.github.io/}{{\textcolor{black}{\underline{Dr.Sivaraman}}}} from \href{https://news.cs.nyu.edu/}{{\textcolor{black}{\underline{Systems@NYU}}}}
        \item \href{https://www.ce.cit.tum.de/en/lkn/team/staff/kellerer-wolfgang/}{{\textcolor{black}{\underline{Prof. Dr.-Ing. Wolfgang Kellerer}}}} from \href{https://www.ce.cit.tum.de/en/lkn/home/}{{\textcolor{black}{\underline{LKN@TUM}}}}
    \end{itemize}
\end{itemize}
The work leverages the current desire to migrate financial exchanges to the public cloud. 
However, the lack of an available cloud-native multicast mechanism still inhibits said shift.
\href{https://arxiv.org/abs/2402.09527}{{\textcolor{black}{\underline{Jasper}}}} presents itself as 
a solution, employing an overlay multicast tree, clock synchronization, and more
to achieve a fair and performant cloud-tenant multicast prototype. 
Beyond aiding in its development, and porting it across cloud platforms such as AWS and GCP, 
the core of my contribution relies on developing a heuristic to better select VMs across Jasper's tree-like network, which is 
being utilized as my final master thesis credit for my original university TUM.
% This project lies at the intersection of finance and computer systems, enabling a future
% with more accessible and efficient trading markets.
% Engaging in a project at the intersection of finance and computer systems, aimed at enhancing
% accessibility and efficiency in trading markets, is profoundly intriguing.
%
% Financial exchanges consider a migration to the cloud for scalability, robustness, and cost-efficiency.
% \href{https://arxiv.org/abs/2402.09527}{{\textcolor{black}{\underline{Jasper}}}} presents a scalable 
% and fair multicast solution for cloud-based exchanges, addressing the lack of cloud-native mechanisms for such. 
% To achieve this, Jasper employs an overlay multicast tree, leveraging clock synchronization, kernel-bypass,  
% and more. My contribution is to develop a heuristic to better select VM's across Jasper's tree network 
% to improve latency and overall performance.
% There is value in providing a foundation for said migration to the cloud, 
% as the barrier of entry to automated or high-frequency trading as a business 
% could be significantly lowered.

% \begin{itemize}
% \item CSCI-GA 3033 - Technologies in Finance - \textit{TDB}
    % \item CSCI-GA 2250 - Operating Systems - \textbf{A}
% \end{itemize}
\divider

\cvevent{M.Sc. Electrical and Computer Engineering}
        {Technical University of Munich}
        {Oct '20' -- {\textcolor{black}{\textbf{Graduation: Sept 2024}}}}{Munich, GER}
\begin{itemize}
\item M.Sc. Thesis {\textcolor{black}{\textit{(Last Credit performed Externally in NY)}}}
\item EI70530: Embedded Systems and Security
\item EI71104: Embedded System Design for Machine Learning
\item EI78039: High Performance Computing for Machine Intelligence
\item EI78014: Secure SoCs for IoT
% \item GPA: 3.0 (American Standards) -- 2.4 (German Standards)
\end{itemize}
\divider

\cvevent{B.Sc. Electrical and Computer Engineering}
        {Technical University of Munich}
        {Feb '17 -- Sept '20}{Munich, GER}
\begin{itemize}
\item \textbf{\textcolor{black}{German GPA: }} 2.2 (Top 37\%) -- \href{https://drive.google.com/file/d/1Xesfn8HF9g4oplwqPZMhcFvGFLuyVzIM/view?usp=sharing}{{\textcolor{black}{\underline{See Grade Distribution}}}}
\item EI06861: Embedded Systems Programming Lab (Tutor)
% \item GPA: 3.0 (American Standards) -- 2.4 (German Standards)
% \item See \href{https://drive.google.com/file/d/1Xesfn8HF9g4oplwqPZMhcFvGFLuyVzIM/view?usp=sharing}{{\textcolor{black}{\underline{Grade Distribution/Explanation}}}}
% \item GPA: 3.1 (American Standards) -- 2.2 (German Standards)
% \item Previously Unfinished EE B.Sc. at PUC-RIO, Brazil -- 2013-2016
\end{itemize}

\cvsection{Certificates \& Misc}
\begin{itemize}
    \item \href{https://courses.edx.org/certificates/f9250573933e4a3e87e8b28ea989bf99}{{\textcolor{black}{UCSD: Data Structures Fundamentals}}}
    \item \href{https://courses.edx.org/certificates/af6115bce0c646aa95f6aaa6c98acb09}{{\textcolor{black}{UT Austin: Embedded Systems - uC I/O}}}
\end{itemize}
% \begin{itemize}
% 	\item \href{https://courses.edx.org/certificates/f9250573933e4a3e87e8b28ea989bf99}{{\textcolor{black}{UCSD: Data Structures Fundamentals}}}
% 	\item \href{https://courses.edx.org/certificates/af6115bce0c646aa95f6aaa6c98acb09}{{\textcolor{black}{UT Austin: Embedded Systems - uC I/O}}}
% 	% \item \href{https://drive.google.com/file/d/1o92biySEgdC3Jg6H8UU-0r6dPcCCKwKe/view?usp=sharing}{{\textcolor{black}{Goethe - C1 German}}}
% \end{itemize}


\cvsection{Personal Projects}

\begin{itemize}
\item \href{https://github.com/duclos-cavalcanti/FreeRTOS-SpaceInvaders}
     {\underline{{\textcolor{black}{FreeRTOS Space Invaders}}} \hfill  \textcolor{black}{[C, RTOS, Multi-Threaded]}}
% \begin{itemize}
% 	\item The work was written in C and done so as a multi-threaded FreeRTOS application. 
% \end{itemize}

 \item \href{https://github.com/duclos-cavalcanti/microsemi-error-detection}
     {\underline{{\textcolor{black}{(16,11) Hamming-Code Err. Detection}}} \hfill  \textcolor{black}{[C, VHDL, FPGA]}}
% \begin{itemize}
%     \item Error detection and correction algorithm implemented on the Microsemi SmartFusion2 SoC.
% \end{itemize}

 \item \href{https://github.com/duclos-cavalcanti/Open-MPI-ValueIteration}
     {\underline{{\textcolor{black}{OpenMPI Value Iteration}}} \hfill  \textcolor{black}{[C++, HPC, Distributed]}}
% \begin{itemize}
% 	\item An asynchronous value iteration model implemented as a dynamic programming problem. Used C++ and OpenMPI.
% \end{itemize}

 \item \href{https://github.com/duclos-cavalcanti/serve}
     {\underline{{\textcolor{black}{Serve}}} \hfill \textcolor{black}{[Golang, CLI, Tooling]}}
% \begin{itemize}
% 	\item Small CLI menu in Golang.
% \end{itemize}
\end{itemize}

% \cvsection{Languages}
%     $\bullet$\ \textbf{\textcolor{black}{German: Fluent}} \hfill $\bullet$\ \textbf{\textcolor{black}{Portuguese: Fluent}} \hfill $\bullet$\ \textbf{\textcolor{black}{Spanish: Beginner}}

% \begin{itemize}
%     % \item \textbf{\textcolor{black}{English: Fluent}}
%     \item \textbf{\textcolor{black}{German: Fluent}}
%     \item \textbf{\textcolor{black}{Portuguese: Fluent}}
% \end{itemize}

% \cvskill{English}{5}
% \divider
%
% \cvskill{German}{5}
% \divider
%
% \cvskill{Portuguese}{5}
% \divider

% \cvskill{Spanish}{2.5}

\medskip

\nocite{*}

\switchcolumn
\cvsection{Experience}
% Full-time research intern at the Real-Time Computer Systems department within TUM. I worked with Google's Edge Coral TPU and benchmarking it's performance. Benchmarking was done through the analysis of USB traffic during model inference via pyshark. Automation tof training, freezing, inference and hardware deployment of several ML Models through Tensorflow was also required.

\cvevent{Research Assistant}{EDA Department - TU Munich}{Jul'22--Oct'22, Oct'20--Mar'21}{Munich, GER}
Two-Part internship, where I was part in developing a Design‑Space‑Exploration framework to optimally 
run the inference of Machine Learning Models across heterogeneous hardware,
including GPUs, CPUs, TPUs and embedded devices. 
One effort consisted of analyzing USB traffic during inference on Google's Coral Edge TPU.

\divider

\cvevent{Embedded Engineer}{Molabo GmbH}{Aug '21 -- Jan '22}{Ottobrunn, GER}
Assisted the \href{https://molabo.com/unternehmen/}{{\textcolor{gray}{\underline{motor-drive team}}}}
as a part-time working student. Responsiblities consisted of
developing streamlined workflows via Jenkinsfiles, CMake and GNU Make, as well as 
developing and unit-testing features for their Embedded/FPGA devices.

\divider

\cvevent{Tutor - Embedded Systems Lab}{RCS Department - TU Munich}{Apr '21 -- Aug '21}{Munich, GER}
Tutor for the Embedded Systems Programming Lab course given at TU Munich.
Aided students regarding their course work and their final project, which 
consisted of writing embedded FreeRTOS applications in C.

\cvsection{Skills}

\cvtag{C++}\cvtag{Linux}\cvtag{Embedded Systems}\cvtag{Cloud}\\
\cvtag{Python}\cvtag{FreeRTOS}\cvtag{IoT}\cvtag{VHDL}\cvtag{FPGAs}
\cvtag{C}\cvtag{TCP/UDP/IP}\cvtag{DPDK}\cvtag{USB}\cvtag{UART}\\
\cvtag{AWS}\cvtag{GCP}\cvtag{Terraform}\cvtag{Docker}\cvtag{Packer}\\
\cvtag{Vagrant}\cvtag{Bash}\cvtag{CMake}\cvtag{Unix}\cvtag{Git}\\
\cvtag{Computer Networking}\cvtag{Operating Systems}\\
\cvtag{High-Perf. Computing}\cvtag{OpenMP}\cvtag{OpenMPI}
\cvtag{Tensorflow}\cvtag{TFLite}\cvtag{TinyML}
\cvtag{CI/CD}\cvtag{Jenkins}\cvtag{Golang}\cvtag{Lua}\cvtag{Rust}\cvtag{JavaScript}
% \cvtag{Golang}\cvtag{Lua}\cvtag{Rust}\cvtag{JavaScript}\cvtag{Latex}\\
% \cvtag{Markdown}\cvtag{HTML}\cvtag{CSS}

\medskip

\cvsection{Strengths}

\cvtag{Hard-working}\cvtag{Detail-Oriented}\cvtag{Inquisitive}\\
\cvtag{Teamworker}\cvtag{Communication}

\end{paracol}
\end{document}

