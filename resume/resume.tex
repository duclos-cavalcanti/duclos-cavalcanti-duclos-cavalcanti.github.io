%-------------------------
% Resume in Latex
% Author : Jake Gutierrez
% Based off of: https://github.com/sb2nov/resume
% License : MIT
%------------------------

\documentclass[letterpaper,11pt]{article}

\usepackage{latexsym}
\usepackage[empty]{fullpage}
\usepackage{titlesec}
\usepackage{marvosym}
\usepackage[usenames,dvipsnames]{color}
\usepackage{verbatim}
\usepackage{enumitem}
\usepackage[hidelinks]{hyperref}
\usepackage{fancyhdr}
\usepackage[english]{babel}
\usepackage{tabularx}
\usepackage{fontawesome5}
\usepackage{multicol}
\setlength{\multicolsep}{-3.0pt}
\setlength{\columnsep}{-1pt}
\input{glyphtounicode}


%----------FONT OPTIONS----------
% sans-serif
% \usepackage[sfdefault]{FiraSans}
% \usepackage[sfdefault]{roboto}
% \usepackage[sfdefault]{noto-sans}
% \usepackage[default]{sourcesanspro}

% serif
% \usepackage{CormorantGaramond}
% \usepackage{charter}


\pagestyle{fancy}
\fancyhf{} % clear all header and footer fields
\fancyfoot{}
\renewcommand{\headrulewidth}{0pt}
\renewcommand{\footrulewidth}{0pt}

% Adjust margins
\addtolength{\oddsidemargin}{-0.6in}
\addtolength{\evensidemargin}{-0.5in}
\addtolength{\textwidth}{1.19in}
\addtolength{\topmargin}{-.7in}
\addtolength{\textheight}{1.4in}

\urlstyle{same}

\raggedbottom
\raggedright
\setlength{\tabcolsep}{0in}

% Sections formatting
\titleformat{\section}{
  \vspace{-4pt}\scshape\raggedright\large\bfseries
}{}{0em}{}[\color{black}\titlerule \vspace{-5pt}]

% Ensure that generate pdf is machine readable/ATS parsable
\pdfgentounicode=1

%-------------------------
% Custom commands
\newcommand{\resumeItem}[1]{
  \item\small{
    {#1 \vspace{-2pt}}
  }
}

\newcommand{\classesList}[4]{
    \item\small{
        {#1 #2 #3 #4 \vspace{-2pt}}
  }
}

\newcommand{\resumeSubheading}[4]{
  \vspace{-2pt}\item
    \begin{tabular*}{1.0\textwidth}[t]{l@{\extracolsep{\fill}}r}
      \textbf{#1} & \textbf{\small #2} \\
      \textit{\small#3} & \textit{\small #4} \\
    \end{tabular*}\vspace{-7pt}
}

\newcommand{\resumeSubSubheading}[2]{
    \item
    \begin{tabular*}{0.97\textwidth}{l@{\extracolsep{\fill}}r}
      \textit{\small#1} & \textit{\small #2} \\
    \end{tabular*}\vspace{-7pt}
}

\newcommand{\resumeProjectHeading}[2]{
    \item
    \begin{tabular*}{1.001\textwidth}{l@{\extracolsep{\fill}}r}
      \small#1 & \textbf{\small #2}\\
    \end{tabular*}\vspace{-7pt}
}

\newcommand{\resumeSubItem}[1]{\resumeItem{#1}\vspace{-4pt}}

\renewcommand\labelitemi{$\vcenter{\hbox{\tiny$\bullet$}}$}
\renewcommand\labelitemii{$\vcenter{\hbox{\tiny$\bullet$}}$}

\newcommand{\resumeSubHeadingListStart}{\begin{itemize}[leftmargin=0.0in, label={}]}
\newcommand{\resumeSubHeadingListEnd}{\end{itemize}}
\newcommand{\resumeItemListStart}{\begin{itemize}}
\newcommand{\resumeItemListEnd}{\end{itemize}\vspace{-5pt}}

%-------------------------------------------
%%%%%%  RESUME STARTS HERE  %%%%%%%%%%%%%%%%%%%%%%%%%%%%


\begin{document}

%----------HEADING----------
% \begin{tabular*}{\textwidth}{l@{\extracolsep{\fill}}r}
%   \textbf{\href{http://sourabhbajaj.com/}{\Large Sourabh Bajaj}} & Email : \href{mailto:sourabh@sourabhbajaj.com}{sourabh@sourabhbajaj.com}\\
%   \href{http://sourabhbajaj.com/}{http://www.sourabhbajaj.com} & Mobile : +1-123-456-7890 \\
% \end{tabular*}

\begin{center}
    {\Huge \scshape Daniel Duclos-Cavalcanti} \\ \vspace{1pt}
    \textbf{\large{\scshape Computer Engineer}} \\ \vspace{1pt}
    55th Street New York, New York 10019 \\ \vspace{1pt}
    \small 516-912-7975 $|$ U.S. Citizen $|$
    \href{mailto:daniel@duclos.dev}{\underline{daniel@duclos.dev}} $|$ 
    \href{https://www.duclos.dev}{\underline{www.duclos.dev}} $|$
    \href{https://www.linkedin.com/in/duclos-cavalcanti/}{\underline{linkedin/duclos-cavalcanti}} $|$
    \href{https://github.com/duclos-cavalcanti}{\underline{github/duclos-cavalcanti}} 
\end{center}

% \section{Summary}
% \small{
% Creative thinker and problem-solver with a masters and bachelors in 
% computer engineering from Germany. Today, I am in New York, 
% collaborating on research with Dr.Sivaraman (NYU) on distributed 
% low-latency networking on the cloud.
% }

\vspace{-16pt}
%-----------EDUCATION-----------
\section{Education}
  \resumeSubHeadingListStart
    %\resumeSubheading
        %{New York University}{Sept 2023 -- May 2024}
        %{Visiting Non-Degree Graduate Student}{New York, USA}
        %\resumeItemListStart
        %    %\resumeItem{M.Sc. Thesis: \textbf{VM Selection Heuristic for Multicast Overlay Trees in the Cloud}}
        %    %\resumeItem{\textbf{GPA 4.0}}
        %    \resumeItem{Co-Authored Publication: \href{https://arxiv.org/abs/2402.09527}{Design and Implementation of A Scalable Financial Exchange in the Cloud}}
        %    \resumeItem{\textbf{Related Coursework}: Operating Systems, Technologies in Finance -- \textbf{GPA 4.0}}
        %\resumeItemListEnd

    \resumeSubheading
        {Technical University of Munich}{Oct 2020 -- Oct 2024}
        {M.Sc. Electrical and Computer Engineering}{Munich, Germany}
        \resumeItemListStart
            %\resumeItem{M.Sc. Thesis: \textbf{VM Selection Heuristic for Multicast Overlay Trees in the Cloud}}
            \resumeItem{Visiting Non-Degree Graduate Student: \textbf{New York University} -- \textbf{GPA 4.0} (2023 -- 2024)}
            %\resumeItem{Co-Authored Publication: \href{https://arxiv.org/abs/2402.09527}{Design and Implementation of A Scalable Financial Exchange in the Cloud}}
            \resumeItem{Master Thesis: \textbf{"VM Selection Heuristic for Financial Exchanges in the Cloud"} -- (\href{https://github.com/duclos-cavalcanti/TreeBuilder}{\textit{TreeBuilder}})}
            \resumeItem{\textbf{Related Coursework}: Operating Systems, Machine Learning Methods, High Performance Computing Lab}
            %\resumeItem{\textbf{Related Coursework}: Machine Learning Methods, HW-SW Co-Design, High Performance Computing Lab}
        \resumeItemListEnd
        % \resumeItemListStart
        %     \resumeItem{M.Sc. Thesis: \textbf{VM Selection Heuristic for Multicast Overlay Trees in the Cloud}}
        %     \resumeItem{Co-Authored Publication: \href{https://arxiv.org/abs/2402.09527}{Design and Implementation of A Scalable Financial Exchange in the Cloud}}
        %     \resumeItem{Visiting Graduate Student: \textbf{New York University}}
        %     \resumeItem{\small{\textbf{Related Coursework}: Operating Systems, Machine Learning Methods, High Performance Computing Lab}}
        % \resumeItemListEnd

    %\vspace{1.0pt}
    \resumeSubheading
        {Technical University of Munich}{Oct 2016 -- Sept 2020}
        {B.Sc. Electrical and Computer Engineering}{Munich, Germany}
  \resumeSubHeadingListEnd

%------RELEVANT COURSEWORK-------
%\section{Relevant Coursework}
%    %\resumeSubHeadingListStart
%        \begin{multicols}{4}
%            \begin{itemize}[itemsep=-5pt, parsep=3pt]
%                \item\small Data Structures
%                \item Software Methodology
%                \item Algorithms Analysis
%                \item Database Management
%                \item Artificial Intelligence
%                \item Internet Technology
%                \item Systems Programming
%                \item Computer Architecture
%            \end{itemize}
%        \end{multicols}
%        \vspace*{2.0\multicolsep}
    %\resumeSubHeadingListEnd

\vspace{-5pt}
\section{Publications}
    \vspace{-5.0pt}
    \resumeSubHeadingListStart
        \resumeProjectHeading
          {\textbf{Design and Implementation of A Scalable Financial Exchange in the Cloud} $|$ \emph{\href{https://arxiv.org/abs/2402.09527}{\underline{(Paper)}}}}{Jan 2024 -- Present}
          \resumeItemListStart
                \resumeItem{Novel Cloud financial exchange achieving low latency of $<=$ 250 µs, with a difference $<$ 1 µs for 1K receivers.}
                \resumeItem{Achieves better scalability and around 50\% lower latency than the multicast service provided by AWS. }
                %\resumeItem{Implemented kernel-bypass techniques (DPDK) to scale performance up to a 35K multicast packet rate.}
                %\resumeItem{Used kernel-bypass techniques (DPDK) to scale and achieve $\sim$50\% lower latency than AWS's multicast service.}
                \resumeItem{Enhanced performance using kernel-bypass (DPDK) to reach up to a 35K multicast packet rate per second.}
          \resumeItemListEnd
    \resumeSubHeadingListEnd

%-----------EXPERIENCE-----------
\vspace{-8pt}
\section{Experience}
    \resumeSubHeadingListStart
        \resumeSubheading
            {Research Assistant}{Jul 2022 -- Oct 2022}
            {TU Munich}{Munich, Germany}
            \resumeItemListStart
                \resumeItem{Collaborated on \href{https://github.com/alxhoff/TensorDSE}{\underline{TensorDSE}}, a Design-Space Exploration framework to accelerate machine learning model deployments.}
                \resumeItem{Assessed the performance metrics of multiple ML models across GPUs, CPUs and TPUs with TensorFlow Lite.}
                \resumeItem{Generated cost analysis reports on Google's Coral Edge TPU via USB traffic analysis (PyShark) during inference.}
                \resumeItem{Established that on average up to 57\% of total inference time consists of data transmission with the external TPU.}
                %\resumeItem{Generated cost analysis reports on Google's Coral Edge TPU via USB traffic analysis (PyShark) during transmission.}
                %\resumeItem{Discovered that over 60\% of total inference time was attributed to data transmission with the external TPU.}
                \resumeItem{TensorDSE consumed reports to map a model's deployment optimally onto an available set of hardware devices.}
            \resumeItemListEnd

        \resumeSubheading
            {Embedded Software Engineer Intern}{Aug 2021 -- Jan 2022}
            {Molabo GmbH}{Ottobrunn, Germany}
            \resumeItemListStart
                %\resumeItem{Added unit tests (GTest), achieving over 25\% test coverage for the team's motor embedded controller critical features.}
                \resumeItem{Increased test coverage (GTest) up to 25\% on safety-critical motor controller features, identifying and resolving bugs.}
                \resumeItem{Developed state simulation tooling with Linux’s virtual CAN interface to validate motor functionality in real-time.}
                \resumeItem{Extended the firmware update system for over 18 clients, enhancing reliability of partial updates via CAN bus.}
                \resumeItem{Automated build and testing workflows via Jenkinsfiles, Makefiles, and CMake, supporting a team of over 10 engineers.}
            \resumeItemListEnd

        \resumeSubheading
            {Tutor (Embedded Systems Programming Lab)}{Apr 2021 -- Aug 2021}
            {TU Munich}{Munich, Germany}
            \resumeItemListStart
                \resumeItem{Mentored over 12 students on designing and developing low-level embedded FreeRTOS applications in C.}
                \resumeItem{Conducted 30+ sessions on best practices in software engineering, concurrency, performance, and real-time scheduling.}
            \resumeItemListEnd
    \resumeSubHeadingListEnd 

% -------------------- SKILLS --------------------
\vspace{-8.0pt}
\section{Technical Skills}
 \begin{itemize}[leftmargin=0.15in, label={}]
    \small{\item{
    \textbf{Languages}{: C++, Python, Golang, Java, C, Bash, JavaScript, HTML, CSS, Lua, VHDL} \\
    \textbf{Cloud Services}{: Google Cloud Platform (GCP), Amazon EC2 (AWS), Terraform, Packer, Vagrant} \\
    \textbf{Tools}{: Linux, Unix Shell, Git, Github CI/CD, Jenkins, CMake, GNU Make, Bazel, Vim, VSCode} \\
    \textbf{Technologies}{: Docker, ZeroMQ, DPDK, MPI, FreeRTOS, FPGA, IoT, TensorFlow, Scipy, NumPy, Pandas, OpenMP} \\
    % \textbf{Technologies}{: Cloud Computing, Computer Networking, Operating Systems, Machine Learning, HPC, FPGAs} \\
    % \textbf{Protocols}{: TCP, UDP, IP, ETHERNET, USB, UART} \\
    % \textbf{Certificates}{: UCSD: Data Structures Fundamentals, UT Austin: Embedded Systems - uC I/O} \\
    % \textbf{Hardware}{: Raspberry PIs, Embedded Linux, ARM Cortex MCUs, USB, TCP, UDP, IP, UART, GPIO} \\
    \textbf{Verbal/Written}{: German -- Fluent, Portuguese -- Fluent}
    }}
 \end{itemize}

\vspace{-8.0pt}
\section{Projects}
    \vspace{-5.0pt}
    \resumeSubHeadingListStart
        \resumeProjectHeading
            {\href{https://github.com/duclos-cavalcanti/TreeBuilder}{\textbf{Cloud-TreeBuilder}} $|$ \emph{GCP, ZMQ, Terraform, Python, C++, Distributed Systems, Heuristic}}{\small{Mar 2024 -- Present}}
            \resumeItemListStart
              %\resumeItem{Launches and selects K out of N VMs in a cluster to create an optimal multicast tree of depth D and fan-out F.}
              %\resumeItem{Deploys UDP based probe jobs on VM subsets, collecting data regarding individual network performance (JSON).}
              %\resumeItem{Applies a developed heuristic on collected data to select VMs for a tree layer by layer. }
              %\resumeItem{Uses UDP probe jobs to examine VMs network performance, selecting instances for a tree layer by layer. }
              %\resumeItem{Uses terraform to manage cloud state, ZMQ for node communication and Protobufs for data serialization.}
              %\resumeItem{Leverages \href{https://www.clockwork.io/clock-sync/}{clockwork's high-precision software clock synchronization daemon} to provide a global common clock.}

            \resumeItem{Optimally selects K out of N VMs in a cluster to form a multicast tree of depth D and fan-out F, minimizing latency.}
            \resumeItem{Deployed UDP-based probe jobs on VMs, gathering network performance data for informed heuristic selection (JSON).}
            \resumeItem{Integrated Terraform for cloud state management, ZMQ for node communication, and Protocol Buffers for serialization.}
            \resumeItem{Improved multicast latency up to 24\% for a cluster of 25 VMs and multicast tree of depth 3 and fan-out 2.}
            \resumeItemListEnd
        \vspace{-16pt}
        \resumeProjectHeading
            {\href{https://github.com/duclos-cavalcanti/Open-MPI-ValueIteration}{\textbf{Open-MPI Value Iteration}} $|$ \emph{C++, Parallel-Computing, MPI, HPC}}{\small{Mar 2022}}
            \resumeItemListStart
              %\resumeItem{Uses MPI techniques to iteratively distribute workload across an HPC cluster and gather results.}
              %\resumeItem{Uses MPI techniques to distribute workload across an HPC cluster to solve a stochastic navigation problem.}
              \resumeItem{An HPC prototype that solves a stochastic navigation problem through Asynchronous Value Iteration (AVI).}
              \resumeItem{Leveraged MPI to iteratively distribute workload across an HPC cluster, executing 52\% faster than in single-threaded.}
            \resumeItemListEnd
        \vspace{-16pt}
        \resumeProjectHeading
            {\href{https://github.com/duclos-cavalcanti/microsemi-error-detection}{\textbf{Hamming Code Error Detection (16,11)}} $|$ \emph{C, VHDL, FPGA, SoC, UART}}{\small{Feb 2021}}
            \resumeItemListStart
              \resumeItem{Implemented an error detection/correction algorithm for packet transmission on Microsemi's SF2 FPGA/SoC.}
              \resumeItem{Error-injected packets sent between host and SoC via UART and offloaded to the FPGA for detection/correction.}
            \resumeItemListEnd
        %\vspace{-16pt}
        %\resumeProjectHeading
        %{\href{https://github.com/duclos-cavalcanti/FreeRTOS-SpaceInvaders}{\textbf{FreeRTOS-SpaceInvaders}} $|$ \emph{C, RTOS, Multi-Threaded}}{\small{Sept 2020}}
        %\resumeItemListStart
        %  \resumeItem{Implemented the famous arcade game as a multi-threaded FreeRTOS application in C.}
        %\resumeItemListEnd

    \resumeSubHeadingListEnd

%
%-----------PROGRAMMING SKILLS-----------
%\section{Technical Skills}
% \begin{itemize}[leftmargin=0.15in, label={}]
%    \small{\item{
%     \textbf{Languages}{: Python, Java, C, HTML/CSS, JavaScript, SQL} \\
%     \textbf{Developer Tools}{: VS Code, Eclipse, Google Cloud Platform, Android Studio} \\
%     \textbf{Technologies/Frameworks}{: Linux, Jenkins, GitHub, JUnit, WordPress} \\
%    }}
% \end{itemize}
% \vspace{-16pt}
%
%
%%-----------INVOLVEMENT---------------
%\section{Leadership / Extracurricular}
%    \resumeSubHeadingListStart
%        \resumeSubheading{Fraternity}{Spring 2020 -- Present}{President}{University Name}
%            \resumeItemListStart
%                \resumeItem{Achieved a 4 star fraternity ranking by the Office of Fraternity and Sorority Affairs (highest possible ranking).}
%                \resumeItem{Managed executive board of 5 members and ran weekly meetings to oversee progress in essential parts of the chapter.}
%                \resumeItem{Led chapter of 30+ members to work towards goals that improve and promote community service, academics, and unity.}
%            \resumeItemListEnd
%        
%    \resumeSubHeadingListEnd


\end{document}
