\documentclass[10pt,a4paper,ragged2e,withhyper]{altacv}
%% AltaCV uses the fontawesome5 and packages.
%% See http://texdoc.net/pkg/fontawesome5 for full list of symbols.

% Change the page layout if you need to
\geometry{left=1.25cm,right=1.25cm,top=1.5cm,bottom=1.5cm,columnsep=1.2cm}

% The paracol package lets you typeset columns of text in parallel
\usepackage{paracol}

% Change the font if you want to, depending on whether
% you're using pdflatex or xelatex/lualatex
\ifxetexorluatex
  % If using xelatex or lualatex:
  \setmainfont{Roboto Slab}
  \setsansfont{Lato}
  \renewcommand{\familydefault}{\sfdefault}
\else
  % If using pdflatex:
  \usepackage[rm]{roboto}
  \usepackage[defaultsans]{lato}
  % \usepackage{sourcesanspro}
  \renewcommand{\familydefault}{\sfdefault}
\fi

% Change the colours if you want to
\definecolor{SlateGrey}{HTML}{2E2E2E}
\definecolor{LightGrey}{HTML}{666666}
\definecolor{TealGreen}{HTML}{5BC8AF}
\colorlet{name}{black}
\colorlet{tagline}{black}
\colorlet{heading}{black}
\colorlet{headingrule}{gray}
\colorlet{subheading}{blue}
\colorlet{accent}{TealGreen}
\colorlet{emphasis}{SlateGrey}
\colorlet{body}{LightGrey}

% Change some fonts, if necessary
\renewcommand{\namefont}{\Huge\rmfamily\bfseries}
\renewcommand{\personalinfofont}{\footnotesize}
\renewcommand{\cvsectionfont}{\LARGE\rmfamily\bfseries}
\renewcommand{\cvsubsectionfont}{\large\bfseries}


% Change the bullets for itemize and rating marker
% for \cvskill if you want to
\renewcommand{\cvItemMarker}{{\small\textbullet}}
\renewcommand{\cvRatingMarker}{\faCircle}
% ...and the markers for the date/location for \cvevent
% \renewcommand{\cvDateMarker}{\faCalendar*[regular]}
% \renewcommand{\cvLocationMarker}{\faMapMarker*}

% If your CV/résumé is in a language other than English,
% then you probably want to change these so that when you
% copy-paste from the PDF or run pdftotext, the location
% and date marker icons for \cvevent will paste as correct
% translations. For example Spanish:
% \renewcommand{\locationname}{Ubicación}
% \renewcommand{\datename}{Fecha}

\begin{document}
\name{Daniel Duclos-Cavalcanti}
\tagline{American/Brazilian -- Electrical and Computer Engineer}

\personalinfo{%
  % Not all of these are required!
  \homepage{www.duclos.dev}
  \email{me@duclos.dev}
  \phone{+1-516-912-7975}
  \location{New York, NY, USA}
  \github{duclos-cavalcanti}
  %\linkedin{daniel-duclos-cavalcanti}
  %% You can add your own arbitrary detail with
  %% \printinfo{symbol}{detail}[optional hyperlink prefix]
  % \printinfo{\faPaw}{Hey ho!}[https://example.com/]
  %% Or you can declare your own field with
  %% \NewInfoFiled{fieldname}{symbol}[optional hyperlink prefix] and use it:
  % \NewInfoField{gitlab}{\faGitlab}[https://gitlab.com/]
  % \gitlab{your_id}
  %%
  %% For services and platforms like Mastodon where there isn't a
  %% straightforward relation between the user ID/nickname and the hyperlink,
  %% you can use \printinfo directly e.g.
  % \printinfo{\faMastodon}{@username@instace}[https://instance.url/@username]
  %% But if you absolutely want to create new dedicated info fields for
  %% such platforms, then use \NewInfoField* with a star:
  % \NewInfoField*{mastodon}{\faMastodon}
  %% then you can use \mastodon, with TWO arguments where the 2nd argument is
  %% the full hyperlink.
  % \mastodon{@username@instance}{https://instance.url/@username}
}

\makecvheader
%% Depending on your tastes, you may want to make fonts of itemize environments slightly smaller
% \AtBeginEnvironment{itemize}{\small}

%% Set the left/right column width ratio to 6:4.
\columnratio{0.6}

% Start a 2-column paracol. Both the left and right columns will automatically
% break across pages if things get too long.
\begin{paracol}{2}
\cvsection{Education}

\cvevent{M.S.\ Computer Science (Exchange Student)}
        {New York University}
        {Sept 2023 -- Now / Aug 2024}{New York, USA}
\begin{itemize}
\item Master Thesis - VM Selection for Financial Exch. in the Cloud    
    \begin{itemize}
        \item Co-Advised by \href{https://anirudhsk.github.io/}{{\textcolor{black}{\underline{Dr.Sivaraman}}}} at \href{https://news.cs.nyu.edu/}{{\textcolor{black}{\underline{Systems@NYU}}}}
        \item \href{https://arxiv.org/abs/2402.09527}{{\textcolor{black}{\underline{Jasper: Fair Multicast for Financial Exch. in the Cloud}}}}
    \end{itemize}
\item CSCI-GA 3033  - Technologies in Finance - \textit{TBD}
\item CSCI-GA 2250  - Operating Systems - \textbf{A}
\end{itemize}

\divider

\cvevent{B.Sc. and M.Sc.\ Electrical and Computer Engineering}
        {Technical University of Munich}
        {B.Sc: Feb '17 -- Sept '20 | M.Sc: Oct '20' -- Now}{Munich, GER}
\begin{itemize}
    \item M.Sc. Thesis {\textcolor{black}{\textit{(Last Credit and Ongoing at NYU)}}}
\item M.Sc. GPA: 3.0 (American Standards), 2.5 (German Standards) 
\item B.Sc. Thesis, 1.3 (German Standards) -- Netlist Error Modeling
\item B.Sc. GPA: 3.1 (American Standards), 2.2 (German Standards)
\item Unfinished Previous B.Sc., PUC-RIO, Brazil -- 2013-2016
\end{itemize}
\divider

\cvsection{Experience}

\cvevent{Assistant Researcher}{EDA Department - TU Munich}{Pt. 2: Jul '22 -- Oct '22 | Pt.1: Oct '20 -- Mar '21}{Munich, GER}
Two part internship where I aided research within the EDA Department. Python Development of a Design‑Space‑Exploration 
framework to find the optimal hardware configuration to run inference of a given ML Model.

\divider

\cvevent{Embedded Engineer}{Molabo GmbH}{Aug '21 -- Jan '22}{Ottobrunn, GER}
Assisted the \href{https://molabo.com/unternehmen/}{{\textcolor{gray}{\underline{motor-drive team}}}}
as a part-time working student, developing for their Embedded and FPGA devices.

\begin{itemize}
\item Supervised pipeline execution on Jenkins/via Jenkinsfiles.
\item Development of containerized unit tests through GoogleTest.
\item Development of embedded code and FPGA Modules.
\item Building toolchains via CMake and GNU Make.
\end{itemize}

\divider

\cvevent{Tutor}{RCS Department - TU Munich}{Apr '21 -- Aug '21}{Munich, GER}
Assisted in teaching the Embedded Systems Programming Lab course given at TU Munich.
Aided students regarding their course work and their final project, which 
consisted of writing embedded FreeRTOS applications in C.

\medskip

\nocite{*}

\switchcolumn

\cvsection{Skills}

\cvtag{C++}\cvtag{Linux}\cvtag{VHDL}\cvtag{FPGA}\cvtag{Python}\\
\cvtag{Bash}\cvtag{CMake}\cvtag{GNU Make}\cvtag{Git}\cvtag{C}\\
\cvtag{Computer Networking}\cvtag{Operating Systems}\\
\cvtag{Embedded Systems}\cvtag{High-Perf. Computing}\\
\cvtag{Docker}\cvtag{TCP/UDP}\cvtag{FreeRTOS}\cvtag{POSIX}\\
\cvtag{Tensorflow}\cvtag{TFLite}\cvtag{TinyML}\cvtag{Protobufs}\\
\cvtag{ZeroMQ}\cvtag{AWS}\cvtag{GCP}\cvtag{Jenkins}\cvtag{Travis CI}\\
\cvtag{OpenMP}\cvtag{OpenMPI}\cvtag{GoogleTest}\cvtag{PyTest}\\
\cvtag{Golang}\cvtag{Lua}\cvtag{Rust}\cvtag{JavaScript}\cvtag{Latex}\\
\cvtag{Markdown}\cvtag{HTML}\cvtag{CSS}

\cvsection{Strengths}

\cvtag{Hard-working}\cvtag{Detail-Oriented}\cvtag{Curious}\\
\cvtag{Teamworker}\cvtag{Great-Communicator}

\cvsection{Languages}

\cvskill{English}{5}
\divider

\cvskill{German}{5}
\divider

\cvskill{Portuguese}{5}
\divider

\cvskill{Spanish}{2.5}

\cvsection{Certificates / Misc.}
\begin{itemize}
	\item \href{https://courses.edx.org/certificates/f9250573933e4a3e87e8b28ea989bf99}{{\textcolor{black}{UCSD: Data Structures Fundamentals}}}
	\item \href{https://courses.edx.org/certificates/af6115bce0c646aa95f6aaa6c98acb09}{{\textcolor{black}{UT Austin: Embedded Systems - uC I/O}}}
	\item \href{https://drive.google.com/file/d/1o92biySEgdC3Jg6H8UU-0r6dPcCCKwKe/view?usp=sharing}{{\textcolor{black}{Goethe - C1 German}}}
\end{itemize}


\cvsection{Personal Projects}

\href{https://github.com/duclos-cavalcanti/FreeRTOS-SpaceInvaders}
     {\textbf{{\textcolor{black}{FreeRTOS Space Invaders}}} \textit{\textcolor{gray}{(Click)}}}
\begin{itemize}
	\item The work was written in C and done so as a multi-threaded FreeRTOS application. 
\end{itemize}

\href{https://github.com/duclos-cavalcanti/microsemi-error-detection}
     {\textbf{{\textcolor{black}{(16,11) Hamming-Code on an FPGA}}} \textit{\textcolor{gray}{(Click)}}}
\begin{itemize}
    \item Error detection and correction algorithm implemented on the Microsemi SmartFusion2 SoC.
\end{itemize}

\href{https://github.com/duclos-cavalcanti/Open-MPI-ValueIteration}
     {\textbf{{\textcolor{black}{OpenMPI Value Iteration}}} \textit{\textcolor{gray}{(Click)}}}
\begin{itemize}
	\item An asynchronous value iteration model implemented as a dynamic programming problem. Used C++ and OpenMPI.
\end{itemize}

\href{https://github.com/duclos-cavalcanti/serve}
     {\textbf{{\textcolor{black}{Serve}}} \textit{\textcolor{gray}{(Click)}}}
\begin{itemize}
	\item Small CLI menu in Golang.
\end{itemize}

\end{paracol}
\end{document}

