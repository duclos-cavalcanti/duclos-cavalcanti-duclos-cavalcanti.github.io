%-------------------------
% Resume in Latex
% Based on: https://github.com/jakegut/resume (which was itself based on https://github.com/sb2nov/resume)
% License : MIT
%------------------------

\documentclass[letterpaper,11pt]{article}

\usepackage{latexsym}
\usepackage[empty]{fullpage}
\usepackage{titlesec}
\usepackage{marvosym}
\usepackage[usenames,dvipsnames]{color}
\usepackage{verbatim}
\usepackage{enumitem}
\usepackage[hidelinks]{hyperref}
\usepackage{fancyhdr}
\usepackage[english]{babel}
\usepackage{tabularx}
\usepackage{xcolor}
\usepackage{fontawesome5}
\usepackage{multicol}

\input{glyphtounicode}

% -------------------- FONT OPTIONS --------------------
% sans-serif
% \usepackage[sfdefault]{roboto}
% \usepackage[sfdefault]{noto-sans}
% serif
% \usepackage{charter}

\pagestyle{fancy}
\fancyhf{} % clear all header and footer fields
\fancyfoot{}
\renewcommand{\headrulewidth}{0pt}
\renewcommand{\footrulewidth}{0pt}

% Adjust margins
\addtolength{\oddsidemargin}{-0.5in}
\addtolength{\evensidemargin}{-0.5in}
\addtolength{\textwidth}{1in}
\addtolength{\topmargin}{-1in} % Default was -.5in
\addtolength{\textheight}{1.0in}

\urlstyle{same}

\raggedbottom
\raggedright
\setlength{\tabcolsep}{0in}

% Section formatting
\titleformat{\section}{
  \vspace{-5pt}\scshape\raggedright\large
}{}{0em}{}[\color{black}\titlerule \vspace{-5pt}]

% Subsection formatting
\titleformat{\subsection}{
  \vspace{-4pt}\scshape\raggedright\large
}{\hspace{-.15in}}{0em}{}[\color{black}\vspace{-8pt}]

% Ensure that generate pdf is machine readable/ATS parsable
\pdfgentounicode=1

% -------------------- CUSTOM COMMANDS --------------------
\newcommand{\resumeItem}[1]{
  \item\small{
    {#1 \vspace{-2pt}}
  }
}

\newcommand{\resumeSubheading}[4]{
  \vspace{-2pt}\item
    \begin{tabular*}{0.97\textwidth}[t]{l@{\extracolsep{\fill}}r}
      \textbf{#1} & #2 \\
      \textit{\small#3} & \textit{\small #4} \\
    \end{tabular*}\vspace{-7pt}
}

\newcommand{\resumeSubSubheading}[2]{
    \item
    \begin{tabular*}{0.97\textwidth}{l@{\extracolsep{\fill}}r}
      \textit{\small#1} & \textit{\small #2} \\
    \end{tabular*}\vspace{-7pt}
}

\newcommand{\resumeProjectHeading}[2]{
    \item
    \begin{tabular*}{0.97\textwidth}{l@{\extracolsep{\fill}}r}
      \small#1 & #2 \\
    \end{tabular*}\vspace{-7pt}
}

\newcommand{\resumeSubItem}[1]{\resumeItem{#1}\vspace{-4pt}}
\newcommand{\resumeSubHeadingListStart}{\begin{itemize}[leftmargin=0.15in, label={}]}
\newcommand{\resumeSubHeadingListEnd}{\end{itemize}}
\newcommand{\resumeItemListStart}{\begin{itemize}}
\newcommand{\resumeItemListEnd}{\end{itemize}\vspace{-5pt}}

\renewcommand\labelitemii{$\vcenter{\hbox{\tiny$\bullet$}}$}

\setlength{\footskip}{4.08003pt}

% -------------------- START OF DOCUMENT --------------------
\begin{document}

% -------------------- HEADING--------------------
\begin{flushright}
  % \vspace{-4pt}
\end{flushright}

\vspace{-5.0pt}

\begin{center}
    \textbf{\Huge \scshape Daniel Duclos-Cavalcanti} \\ \vspace{1pt}
    \textbf{\large{\scshape Computer Engineer}} \\ \vspace{1pt}
    \small 516-912-7975 $|$ New York, NY $|$ U.S. Citizen $|$
    \href{mailto:me@duclos.dev}{\underline{me@duclos.dev}} $|$ 
    \href{https://www.duclos.dev}{\underline{www.duclos.dev}} $|$
    \href{https://www.linkedin.com/in/duclos-cavalcanti/}{\underline{linkedin}} $|$
    \href{https://github.com/duclos-cavalcanti}{\underline{github}} 
\end{center}

\vspace{-8.0pt}

\section{Summary}
\small{
Creative thinker and problem-solver with a masters and bachelors in 
computer engineering from Germany. Today, I am in New York, 
collaborating on research with Dr.Sivaraman (NYU) on distributed 
low-latency networking on the cloud.
}

\vspace{-8.0pt}

% -------------------- SKILLS --------------------
\section{Technical Skills}
 \begin{itemize}[leftmargin=0.15in, label={}]
    \small{\item{
    \textbf{Languages}{: C++, Python, Golang, Rust, C, Bash, JavaScript, HTML, CSS, Lua, VHDL} \\
    \textbf{Tools}{: Linux, Terraform, Google Cloud (GCP), AWS, Docker, Packer, Git, Unix Shell, Makefile, CMake, Vim, VSCode} \\
    \textbf{Frameworks}{: ZeroMQ, DPDK, MPI, Protobufs, Pydantic, Tensorflow, Scipy, Numpy, Pandas, Jenkins, Travis CI} \\
    \textbf{Technologies}{: Cloud Computing, Computer Networking, Operating Systems, Machine Learning, HPC, FPGAs} \\
    % \textbf{Protocols}{: TCP, UDP, IP, ETHERNET, USB, UART} \\
    \textbf{Certificates}{: UCSD: Data Structures Fundamentals, UT Austin: Embedded Systems - uC I/O} \\
    % \textbf{Hardware}{: Raspberry PIs, Embedded Linux, ARM Cortex MCUs, USB, TCP, UDP, IP, UART, GPIO} \\
    \textbf{Verbal/Written}{: German -- Fluent, Portuguese -- Fluent}
    }}
 \end{itemize}

\vspace{-14.0pt}

% -------------------- EXPERIENCE --------------------
\section{Experience}
    \resumeSubHeadingListStart
        \resumeSubheading
            {Research Assistant}{Jul 2022 -- Oct 2022}
            {TU Munich}{Munich, Germany}
            \resumeItemListStart
                \resumeItem{Worked on \href{https://github.com/alxhoff/TensorDSE}{\underline{TensorDSE}}, a Design-Space Exploration framework to guide machine learning model deployments.}
                \resumeItem{Evaluated the performance of various ML models across GPUs, CPUs and TPUs with TensorFlow Lite.}
                \resumeItem{Generated cost analysis reports for Google's Coral Edge TPU via USB traffic analysis (PyShark) during inference.}
                \resumeItem{TensorDSE used reports to distribute a model's inference/deployment onto a set of available hardware devices.}
            \resumeItemListEnd
            \vspace{2.0pt}

        \resumeSubheading
            {Embedded Engineer Intern}{Aug 2021 -- Jan 2022}
            {Molabo GmbH}{Ottobrunn, Germany}
            \resumeItemListStart
                \resumeItem{Added unit-tests (GTest) and code coverage (lcov) to safety critical features of their motor's embedded controller.}
                \resumeItem{Developed tooling for state simulations of their electric motor via Linux's virtual CAN interface and mock APIs.}
                \resumeItem{Extended their firmware update system used by 20+ clients, consisting of partial updates via CAN bus.}
                \resumeItem{Automated build and testing workflows via Jenkinsfiles, Make and CMake for a team of over 10 engineers.}
            \resumeItemListEnd

        % \resumeSubheading
        %     {Tutor - Embedded Systems Programming Lab}{Apr 2021 -- Aug 2021}
        %     {TU Munich}{Munich, Germany}
        %     \resumeItemListStart
        %         % \resumeItem{\small{Tutor for the Embedded Systems Programming Lab at TU Munich.}}
        %         \resumeItem{\small{Supervised and aided 20+ students on their final embedded FreeRTOS labroratory projects in C.}}
        %     \resumeItemListEnd
    \resumeSubHeadingListEnd 

\vspace{-8.0pt}

% -------------------- PROJECTS --------------------
\section{Projects}
    \resumeSubHeadingListStart
        \resumeProjectHeading
            {\href{https://github.com/duclos-cavalcanti/master-arbeit}{\textbf{Cloud TreeFinder}} $|$ \emph{GCP, Terraform, Python, C++, Distributed Systems, ZMQ, Protobufs}}{\small{March 2024 -- Present}}
            \resumeItemListStart
              \resumeItem{Launches a cloud cluster and from a pool of N VMs, creates an optimal multicast tree of depth D and fanout F.}
              \resumeItem{Deploys probe jobs on randomly selected node subsets, collecting and processing resulting reports (JSON). }
              \resumeItem{Applies a developed heuristic from the collected data to select nodes in the tree layer by layer. }
              \resumeItem{Uses terraform to manage cloud state, ZMQ for node communication and Protobufs for data (de)-serialization.}
              % \resumeItem{Uses \href{https://www.clockwork.io/clock-sync/}{clockwork's high-precision software clock synchronization daemon} to provide a global common clock.}
            \resumeItemListEnd
        \resumeProjectHeading
            {\href{https://github.com/duclos-cavalcanti/Open-MPI-ValueIteration}{\textbf{Open-MPI Value Iteration}} $|$ \emph{C++, Multi-Threaded, HPC, MPI}}{\small{Sept 2021 -- Feb 2022}}
            \resumeItemListStart
              \resumeItem{Asynchronous value iteration model to distribute workload on an HPC cluster.}
            \resumeItemListEnd
        % \resumeProjectHeading
        %     {\href{https://github.com/duclos-cavalcanti/microsemi-error-detection}{\textbf{Hamming Code Error Detection}} $|$ \emph{C, VHDL, FPGA, SoC}}{\small{Oct 2022 -- March 2023}}
        %     \resumeItemListStart
        %       \resumeItem{Error detection/correction algorithm for packet transmission on Microsemi's SF2 FPGA/SoC.}
        %     \resumeItemListEnd
        % \resumeProjectHeading
        %     {\href{https://github.com/duclos-cavalcanti/FreeRTOS-SpaceInvaders}{\textbf{FreeRTOS-SpaceInvaders}} $|$ \emph{C, RTOS, Multi-Threaded}}{\small{Aug 2020 -- March 2021}}
        %     \resumeItemListStart
        %       \resumeItem{Implemented the famous arcade game as a multi-threaded FreeRTOS application in C.}
        %     \resumeItemListEnd

    \resumeSubHeadingListEnd

\section{Publications}
    \resumeSubHeadingListStart
        \resumeProjectHeading
          {\textbf{Design and Implementation of A Scalable Financial Exchange in the Cloud} $|$ \emph{\href{https://arxiv.org/abs/2402.09527}{\underline{(Paper)}}}}{Jan 24 -- Present}
          \resumeItemListStart
                \resumeItem{Cloud financial exchange achieving low latency of $<=$ 250 µs and a latency difference $<$ 1 µs, for 1K receivers.}
                \resumeItem{Achieves better scalability and around 50\% lower latency than the multicast service provided by AWS. }
                \resumeItem{Used kernel-bypass techniques (DPDK) to scale performance up to a 35K multicast packet rate.}
          \resumeItemListEnd
    \resumeSubHeadingListEnd

% \vspace{-20.0pt}
\vspace{-8.0pt}

    
% -------------------- EDUCATION --------------------
\section{Education}
  \resumeSubHeadingListStart
    \resumeSubheading
        {New York University: Courant Institute of Mathematical Sciences}{Sept 2023 -- May 2024}
        {Computer Science - Visiting Non-Degree Graduate Student}{GPA 4.0}
        \resumeItemListStart
            \resumeItem{Co-Authored Publication: \href{https://arxiv.org/abs/2402.09527}{Design and Implementation of A Scalable Financial Exchange in the Cloud}}
            \resumeItem{\textbf{Related Coursework}: Operating Systems, Technologies in Finance}
        \resumeItemListEnd

    \vspace{3.0pt}
    \resumeSubheading
        {Technical University of Munich}{Oct 2020 -- \textbf{Oct 2024}}
        {M.Sc. Electrical and Computer Engineering}{}

        \resumeItemListStart
            \resumeItem{M.Sc. Thesis: \textbf{VM Selection Heuristic for Multicast Overlay Trees in the Cloud}}
            \resumeItem{\textbf{Related Coursework}: Machine Learning Methods and Tools, Embedded Design for Machine Learning, \\Chips Multicore Processors, Secure SoCs for IoT, High Performance Computing for Machine Intelligence}
        \resumeItemListEnd

    \vspace{3.0pt}
    \resumeSubheading
        {Technical University of Munich}{Oct 2016 -- Sept 2020}
        {B.Sc. Electrical and Computer Engineering}{}
        % \resumeItemListStart
            % \resumeItem{\textbf{{German GPA: }} 2.2 (Top 37\%) -- \href{https://drive.google.com/file/d/1Xesfn8HF9g4oplwqPZMhcFvGFLuyVzIM/view?usp=sharing}{{\underline{See Grade Distribution}}}}
            % \resumeItem{\textbf{Coursework}: Embedded Systems Programming Lab, Computer Networks, Data Structures}
        % \resumeItemListEnd

    % \vspace{3.0pt}
    % \resumeSubheading
    %     {PUC-Rio}{Jan 2013 -- Sept 2016}
    %     {B.Sc. Electrical Engineering}{Rio de Janeiro, Brazil}
        
  \resumeSubHeadingListEnd

\vspace{-8.0pt}

    
% \section{Certificates}
% \begin{minipage}{0.45\textwidth}
% \begin{itemize}
%     \item UCSD: Data Structures Fundamentals
% \end{itemize}
% \end{minipage}
% \hfill
% \begin{minipage}{0.45\textwidth}
% \begin{itemize}
%     \item UT Austin: Embedded Systems - uC I/O
% \end{itemize}
% \end{minipage}

\end{document}
